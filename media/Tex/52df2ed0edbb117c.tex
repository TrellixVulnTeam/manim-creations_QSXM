\documentclass[preview]{standalone}

\usepackage[english]{babel}
\usepackage[utf8]{inputenc}
\usepackage[T1]{fontenc}
\usepackage{lmodern}
\usepackage{amsmath}
\usepackage{amssymb}
\usepackage{dsfont}
\usepackage{setspace}
\usepackage{tipa}
\usepackage{relsize}
\usepackage{textcomp}
\usepackage{mathrsfs}
\usepackage{calligra}
\usepackage{wasysym}
\usepackage{ragged2e}
\usepackage{physics}
\usepackage{xcolor}
\usepackage{microtype}
\DisableLigatures{encoding = *, family = * }
\linespread{1}

\begin{document}

\begin{center}
Is a notational system for specifying a series of notes to be played simultaneously, written as:
  \[
	  \overline{r} \mid x_{1} , x_{2} , \dotsc  , x_{n - 1} , x_{n}
  \]
  \begin{itemize}
	\item Where $ \overline{r} \in \mathbb{W}$ is the root tone 
	\item  $ x_{1} , x_{2} , \dotsc  , x_{n - 1} , x_{n}$ are a series of intervals that generate the notes
	  \[
	  \overline{r  +  x_{1}}, \overline{r  +  x_{2}}, \ldots, \overline{r  +  x_{n}}, 
	  \]
	  These notes form the chord.
  \end{itemize}
  \subsubsection*{Examples}
  \begin{itemize}
	\item $ \overline{0} \mid 0, 4, 7, 11$ generates the notes $ \overline{0}, \overline{4}, \overline{7}, \overline{11} \leftrightarrow C, E, G, B$ which is a C major 7th chord.
	\item $ \overline{5} \mid 0, 3, 7$ generates the notes $ \overline{5}, \overline{8}, \overline{0} \leftrightarrow F, Ab, C$ which is a F minor triad
  \end{itemize}
\end{center}

\end{document}
