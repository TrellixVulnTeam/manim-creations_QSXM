\documentclass[preview]{standalone}

\usepackage[english]{babel}
\usepackage[utf8]{inputenc}
\usepackage[T1]{fontenc}
\usepackage{lmodern}
\usepackage{amsmath}
\usepackage{amssymb}
\usepackage{dsfont}
\usepackage{setspace}
\usepackage{tipa}
\usepackage{relsize}
\usepackage{textcomp}
\usepackage{mathrsfs}
\usepackage{calligra}
\usepackage{wasysym}
\usepackage{ragged2e}
\usepackage{physics}
\usepackage{xcolor}
\usepackage{microtype}
\DisableLigatures{encoding = *, family = * }
\linespread{1}

\begin{document}

\begin{center}
The interval from  $ \overline{x}$ to $ \overline{y }$ (where $ \overline{x}, \overline{y} \in \mathbb{W}$) is the number of semitones you have to add to $ \overline{x}$ to get to $ \overline{y}$, 
  \[
	\overline{x}  +  \left( y  -  x \right) = \overline{x  +  y  -  x} = \overline{y}
  \]
  Therefore in general the interval from $ \overline{x}$  to $ \overline{y}$ is $y  - x$ . And we define 
  \[
	I\left( \overline{x}, \overline{y}\right) \stackrel{\mathtt{D}}{=} (y  -  x) 
  \]

  For example: $I\left( \overline{4}, \overline{9}\right) = 9  -  4 = 5$
\end{center}

\end{document}
