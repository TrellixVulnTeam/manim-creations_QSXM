\documentclass[preview]{standalone}

\usepackage[english]{babel}
\usepackage[utf8]{inputenc}
\usepackage[T1]{fontenc}
\usepackage{lmodern}
\usepackage{amsmath}
\usepackage{amssymb}
\usepackage{dsfont}
\usepackage{setspace}
\usepackage{tipa}
\usepackage{relsize}
\usepackage{textcomp}
\usepackage{mathrsfs}
\usepackage{calligra}
\usepackage{wasysym}
\usepackage{ragged2e}
\usepackage{physics}
\usepackage{xcolor}
\usepackage{microtype}
\DisableLigatures{encoding = *, family = * }
\linespread{1}

\begin{document}

\begin{center}
It may be seen that if you play on fret $n$ of $\overline{4_L}$, and then fret $n$ of $\overline{9}$, that the second note is also 5 semitones higher, further it can be seen that this holds for any other two consecutively fretted notes, except for one going from string $\overline{7}$ to $\overline{11}$, because this one will have a gap size of 4 semitones.
\end{center}

\end{document}
