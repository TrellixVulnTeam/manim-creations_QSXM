\documentclass[preview]{standalone}

\usepackage[english]{babel}
\usepackage[utf8]{inputenc}
\usepackage[T1]{fontenc}
\usepackage{lmodern}
\usepackage{amsmath}
\usepackage{amssymb}
\usepackage{dsfont}
\usepackage{setspace}
\usepackage{tipa}
\usepackage{relsize}
\usepackage{textcomp}
\usepackage{mathrsfs}
\usepackage{calligra}
\usepackage{wasysym}
\usepackage{ragged2e}
\usepackage{physics}
\usepackage{xcolor}
\usepackage{microtype}
\DisableLigatures{encoding = *, family = * }
\linespread{1}

\begin{document}

\begin{center}
["Let's say $\\mathcal{K} =  \\overline{0} \\mid 0 2 4 5 7 9 11$ (C major) and we are looking at the RIC $\\overline{0} \\mid 0 4 7 11$ .", "Now we change keys and we're now in the key of $ \\mathcal{K} =  \\overline{9} \\mid 0 2 4 5 7 9 11$ (A major), to specify the same kind of interval collection (one that starts on the key's root) we would write $ \\overline{9} \\mid 0 4 7 11$.This notation doesn't really convey what's going on. In this situation $ \\overline{9} \\mid 0 4 7 11$ is going to have the same effect as $ \\overline{0} \\mid 0 4 7 11$ as they're both a chord which is constructed of the same intervals from the root of the key. ", 'To be able to support this type of equality we will develop a new definition.']
\end{center}

\end{document}
