\documentclass[preview]{standalone}

\usepackage[english]{babel}
\usepackage[utf8]{inputenc}
\usepackage[T1]{fontenc}
\usepackage{lmodern}
\usepackage{amsmath}
\usepackage{amssymb}
\usepackage{dsfont}
\usepackage{setspace}
\usepackage{tipa}
\usepackage{relsize}
\usepackage{textcomp}
\usepackage{mathrsfs}
\usepackage{calligra}
\usepackage{wasysym}
\usepackage{ragged2e}
\usepackage{physics}
\usepackage{xcolor}
\usepackage{microtype}
\DisableLigatures{encoding = *, family = * }
\linespread{1}

\begin{document}

\begin{center}
Let's do another example, let's generate the line of the 16th fret with an anchor point on string $\overline{11}$, we can start by realizing that $ 16  \equiv 4 \;(\bmod\; 12) $, so we just have to add 4 semitones to the open string to find the new note. So that note would be $\overline{3}$, and this time only work with the intervals.
\end{center}

\end{document}
