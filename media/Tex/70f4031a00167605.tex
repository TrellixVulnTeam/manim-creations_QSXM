\documentclass[preview]{standalone}

\usepackage[english]{babel}
\usepackage[utf8]{inputenc}
\usepackage[T1]{fontenc}
\usepackage{lmodern}
\usepackage{amsmath}
\usepackage{amssymb}
\usepackage{dsfont}
\usepackage{setspace}
\usepackage{tipa}
\usepackage{relsize}
\usepackage{textcomp}
\usepackage{mathrsfs}
\usepackage{calligra}
\usepackage{wasysym}
\usepackage{ragged2e}
\usepackage{physics}
\usepackage{xcolor}
\usepackage{microtype}
\DisableLigatures{encoding = *, family = * }
\linespread{1}

\begin{document}

\begin{center}
First we can see that $I(\overline{4}, \overline{9}) = 5$, and that the notes produced by playing the $n$-th fret on $\overline{4}_L$ would be $\overline{4 + n}$ , and then the $n$-th fret on the $\overline{9}$ string would be $ \overline{9 + n}$.
\end{center}

\end{document}
